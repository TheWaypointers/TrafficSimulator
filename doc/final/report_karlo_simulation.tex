 \documentclass[a4paper,12pt]{article}
 \usepackage{float}
 \usepackage{hyperref}
 \usepackage{graphicx}
 
 \usepackage{listings}
 \usepackage{color}
 
 \definecolor{dkgreen}{rgb}{0,0.6,0}
 \definecolor{gray}{rgb}{0.5,0.5,0.5}
 \definecolor{mauve}{rgb}{0.58,0,0.82}
 
 \lstset{frame=tb,
 	language=Java,
 	aboveskip=3mm,
 	belowskip=3mm,
 	showstringspaces=false,
 	columns=flexible,
 	basicstyle={\small\ttfamily},
 	numbers=none,
 	numberstyle=\tiny\color{gray},
 	keywordstyle=\color{blue},
 	commentstyle=\color{dkgreen},
 	stringstyle=\color{mauve},
 	breaklines=true,
 	breakatwhitespace=true,
 	tabsize=3
 }
 
 \title{Final report}
 \author{The Waypointers}

\begin{document}

\section{Simulation}

\subsection{about simulation}
The Simulation is the back end of our Traffic Simulator. This runs on a thread and calculates the state of the Traffic Simulator for every n time. It builds the initial state of the simulation using the „WorldState“ object passed on the creation of this class. 
\newline
The „WorldState“ object represents the current state of the simulation. The first world state object with it's roads and junctions gets created on the start of the application. The „TrafficSimulatorStarter“ is the main class in which up on start it creates and passes the XML path to the „MapWorldStateFactory“.
This class then finds the XML, parses trough it creating the „WorldState“ object and returns it to the „TrafficSimulatorStarter“.
\newline
The XML files are used to easier create road networks. But for the "Simulation" to work the XML files must follow certain rules.
\begin{itemize}
	\item The Exit nodes names must start with an 'E'
	\item Directions of the roads must be correct
	\item Junctions must be correctly connected to each other
	\item Traffic light junctions must be added as an XML item, other junctions are normal junctions
\end{itemize}
This a simple example of an XML with 4 roads and 1 traffic light junction:
\begin{lstlisting}
<?xml version="1.0"?>
<simulation>
	<roads>
		<road>
			<origin>A</origin>
			<destination>E1</destination>
			<length>300</length>
			<direction>Left</direction>
		</road>
		<road>
			<origin>A</origin>
			<destination>E2</destination>
			<length>300</length>
			<direction>Up</direction>
		</road>
		<road>
			<origin>A</origin>
			<destination>E3</destination>
			<length>300</length>
			<direction>Right</direction>
		</road>
		<road>
			<origin>A</origin>
			<destination>E4</destination>
			<length>300</length>
			<direction>Down</direction>
		</road>
	</roads>
	<junctions>
		<junction>
			<name>A</name>
			<type>JunctionTrafficLights</type>
		</junction>
	</junctions>
</simulation>
\end{lstlisting}
\subsection{starting the simulation}
Once the initial „WorldState“ object is created,  „TrafficSimulatorStarter“  creates a new „Simulation“ class and passes „WorldState“  to it. „Simulation“ then builds the road network for itself from that object. It contains a weighted graph which represents the road network. Each node equals a junction and each edge equals one side of the road. The vehicles are stored in a hash map where the edge of the graph is the key, and objects are array lists of vehicles. So that for each road we can store more than one vehicle.
\newline
When we start the simulation it starts a while loop inside which every n seconds the „Simulation“ class calculates a new world state for the time passed. We can manipulate the time and the time step using the GUI. It is also possible to pause or restart the simulation trough the GUI.

\begin{lstlisting}
while (true) {
	if (!isSleep) {
		worldState = simulation.getNextState(simulationTimeStep);
		output.NewStateReceived(worldState);
	}
	try {
		Thread.sleep(timeStep);
	} catch (InterruptedException e) {
		e.printStackTrace();
	}
}
\end{lstlisting}

\subsection{"getNextState"}
Every time „Simulation“ calls „getNextState“ it has to go trough three steps: attemption to spawn vehicles, attemption to change the traffic lights color and calculating the new positions of vehicles.\newline
\subsubsection{Junctions}
There are 3 types of Junctions: traffic light junction, normal junction and the exit node. On every new state „Simulation“ tries to change the traffic light color, this happens only if the required time was met. We implement this by making a traffic light step counter, so the color changes on every n state. Normal junctions differ from the traffic light junctions in means that the vehicle has to give the advantage to the vehicle on it's right. The exit node is the point where vehicles enter and leave the road network.
\newline
\subsubsection{New vehicles}
Once the „Simulation“ tries to create  a new vehicle it has a choice of 4 types of vehicles. Those are: reckless car, cautious car, normal car and emergency service. The difference between cars is that the reckless car drives faster than normal and cautious slower. The emergency service is different in a way that it can pass trough the red traffic light and the cars have stop so it can go trough.
The Simulation decides which of these to create by using the „VehicleSpawnRatio“. It gives the ratio of chance of each of the types to spawn, we can change this ratios in the GUI.
\newline
The entering and leaving exit nodes are decided randomly for every vehicle.
\newline
\subsubsection{Moving vehicles}
After the vehicle spawning and changing of traffic lights „Simulation“ calculates the new position of each of the vehicles on the road network. It does this by iterating trough the vehicles and calling the „calculateNextPosition“ method on them. Each of the vehicle has to calculate their own speed, check if there is a vehicle in front of them blocking their path and then check if they are reaching a node(junction). Once the vehicle reaches the junction it has  a different set of rules depending on the type of the junction. For example for traffic light junctions they need to check if the traffic light is green to enter the junction. Once they know they can enter it, they call the „canGoTroughJunction“ method to avoid collisions with other vehicles in junctions. Each vehicle has to check: if the junction is blocked, if the vehicles inside junctions are going to collide with them, if they are turning left so they can signal the other vehicles that they should pay attention not to hit them and they need to check if the vehicle from the opposite side of the road is turning left.
\newline
We have implemented the blocked junction check because dead locks sometimes happen on the normal junctions which are simpler, because if there is a large amount of vehicles and vehicles are trying to enter the junction from each side of the road they don't know who goes first. In this case the car with the advantage is the one that reached the junction first.
\newline
Once each of the vehicles has calculated it's new position the „Simulation“ updates the world state with the new information. It builds a new „VehicleDTO“ class from the common part for each of the vehicles inside Traffic Simulator, it passes them information about the vehicles like road, location vehicle type, speed.




\end{document}