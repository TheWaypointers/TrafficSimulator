\documentclass[a4paper,12pt]{article}
\usepackage{float}
\usepackage{longtable,tabu}
\usepackage{etoc}
\usepackage{listings}
\usepackage[usenames,dvipsnames]{color}
\usepackage[colorlinks=true,linkcolor=blue]{hyperref} 

\definecolor{codegreen}{rgb}{0,0.6,0}
\definecolor{codegray}{rgb}{0.5,0.5,0.5}
\definecolor{codepurple}{rgb}{0.58,0,0.82}
\definecolor{backcolour}{rgb}{0.95,0.95,0.92}

\lstdefinestyle{customasm}{
  backgroundcolor=\color{backcolour},   
  commentstyle=\color{codegreen},
  keywordstyle=\color{magenta},
  numberstyle=\tiny\color{codegray},
  stringstyle=\color{codepurple},
  basicstyle=\footnotesize\ttfamily,
  breakatwhitespace=false,         
  breaklines=true,                 
  captionpos=b,                    
  keepspaces=true,                 
  numbers=left,                    
  numbersep=5pt,                  
  showspaces=false,                
  showstringspaces=false,
  showtabs=false,                  
  tabsize=2,
  linewidth=14cm,
  language=Java
}       

\title{Final report}
\author{The Waypointers}

\begin{document}

%%%%%%%%%%%%%%%%%%%%%%%%%%%%%%%%%%%%%%%%%
% University Assignment Title Page 
% LaTeX Template
%
% This template has been downloaded from:
% http://www.LaTeXTemplates.com
%
% Original author:
% WikiBooks (http://en.wikibooks.org/wiki/LaTeX/Title_Creation)
%
% License:
% CC BY-NC-SA 3.0 (http://creativecommons.org/licenses/by-nc-sa/3.0/)i

\begin{titlepage}

\newcommand{\HRule}{\rule{\linewidth}{0.5mm}} % Defines a new command for the horizontal lines, change thickness here

\center % Center everything on the page
 
%----------------------------------------------------------------------------------------
%	HEADING SECTIONS
%----------------------------------------------------------------------------------------

\textsc{\LARGE King's College London}\\[1.5cm] % Name of your university/college
\textsc{\Large 7CCSMGPR Group Project}\\[0.5cm] % Major heading such as course name
%\textsc{\large Minor Heading}\\[0.5cm] % Minor heading such as course title

%----------------------------------------------------------------------------------------
%	TITLE SECTION
%----------------------------------------------------------------------------------------

\HRule \\[0.4cm]
{ \huge \bfseries Final project report}\\[0.4cm] % Title of your document
\HRule \\[1.5cm]
 
%----------------------------------------------------------------------------------------
%	AUTHOR SECTION
%----------------------------------------------------------------------------------------
\begin{flushleft}
{\Large \emph{Team:}\\
\textsc{The Waypointers}\\}
Haipei Liu\\
Karlo Santini\\
Michal Szewczak\\
Mengzhu Wang\\
Minghao Zhu\\[3cm]
\end{flushleft}
%----------------------------------------------------------------------------------------
%	DATE SECTION
%----------------------------------------------------------------------------------------

{\large \today}\\[3cm] % Date, change the \today to a set date if you want to be precise

%----------------------------------------------------------------------------------------
%	LOGO SECTION
%----------------------------------------------------------------------------------------

%\includegraphics{Logo}\\[1cm] % Include a department/university logo - this will require the graphicx package
 
%----------------------------------------------------------------------------------------

\vfill % Fill the rest of the page with whitespace

\end{titlepage}

%%%%%%%%%%%%%%%%%%%%%%%%%%%%%%%%%%%%%%%%%%
\etocdepthtag.toc{mtchapter}
\etocsettagdepth{mtchapter}{subsection}
\etocsettagdepth{mtappendix}{section}
\tableofcontents

\newpage

\section{Introduction}

\section{Review of related work}

\section{Requirements}

\section{Design}

\section{Implementation}

\section{Teamwork}

\subsection{Tools and libraries used}

\subsubsection*{Tools and applications}

\begin{itemize}
	\item Git\footnote{\url{https://git-scm.com/}} - version control
	\item GitHub\footnote{\url{https://github.com/}} - Git repository management, task assignment, teamwork
	\item \LaTeX\footnote{\url{https://www.latex-project.org/}} - typesetting system, for creating documentation
	\item Gliffy\footnote{\url{https://www.gliffy.com/}} - a Chrome app for creating diagrams
	\item Apache Maven\footnote{\url{https://maven.apache.org/}} - a build tool for building the project easily and for dependency management
	\item Travis CI\footnote{\url{https://travis-ci.org/}} - a continuous integration tool that integrates into GitHub and builds every commit pushed to the repository (in our case - using Maven), also checking if the tests pass
	\item IntelliJ IDEA\footnote{\url{https://www.jetbrains.com/idea/}} - a cross-platform IDE for development in Java (and more). To minimize potential hassles with different development environments, we all agreed on using the same IDE.
\end{itemize}


A significant fact to note that is our team used all the major operating systems (Windows, OS X, Linux), so we focused on choosing cross-platform tools. The only troubles that arised from that were minor layout differences in the Swing GUI of our system.

\subsubsection*{Libraries}

\begin{itemize}
	\item JDK 1.8\footnote{\url{http://www.oracle.com/technetwork/java/javase/downloads/index.html}} - We had several reasons for choosing Java: \begin{itemize}
		\item Familiarity of team members with it
		\item Available on all platforms
		\item Good capabilities of cross-platform GUI
		\item Strong typing allowing for easier maintaining of code integrity
	\end{itemize}
	We chose version 8 because it introduces many much-needed improvements for the language like lambda functions or the Stream API.
	\item JUnit 4.12\footnote{\url{http://junit.org/junit4/}} - The ``de facto'' unit testing library for Java. Good integration with IntelliJ IDEA and Maven.
	\item FEST-Assert 2.0M5\footnote{\url{https://github.com/alexruiz/fest-assert-2.x}} - A library to make assertions in unit tests more readable and easier to compose.
	\item JGraphT 0.7.3 \footnote{\url{http://jgrapht.org/}} - A graph library to take advantage of graph representation of the road network, mainly pathfinding for vehicles.
	\item XStream 1.2.2 \footnote{\url{http://x-stream.github.io/}} - An XML serialization library to bring reading and writing XML road maps to life.
\end{itemize}



\section{Evaluation}

\section{Peer assessment}

\newpage
\appendix

\etocdepthtag.toc{mtappendix}
\etocsettagdepth{mtchapter}{none}
\etocsettagdepth{mtappendix}{subsection}
\tableofcontents‎‎

\section{Git log}

% \input{gitlog.tex}

\section{Main source code}


\newpage
\subsection{Bootstrapper.java}
\lstinputlisting[style=customasm]{../../src/main/java/thewaypointers/trafficsimulator/Bootstrapper.java}
\newpage
\subsection{common/Direction.java}
\lstinputlisting[style=customasm]{../../src/main/java/thewaypointers/trafficsimulator/common/Direction.java}
\newpage
\subsection{common/ExitNodeDTO.java}
\lstinputlisting[style=customasm]{../../src/main/java/thewaypointers/trafficsimulator/common/ExitNodeDTO.java}
\newpage
\subsection{common/helpers/FirstVersionProvider.java}
\lstinputlisting[style=customasm]{../../src/main/java/thewaypointers/trafficsimulator/common/helpers/FirstVersionProvider.java}
\newpage
\subsection{common/helpers/InitialParameters.java}
\lstinputlisting[style=customasm]{../../src/main/java/thewaypointers/trafficsimulator/common/helpers/InitialParameters.java}
\newpage
\subsection{common/helpers/JunctionTestProvider.java}
\lstinputlisting[style=customasm]{../../src/main/java/thewaypointers/trafficsimulator/common/helpers/JunctionTestProvider.java}
\newpage
\subsection{common/helpers/RoadNetworkProvider.java}
\lstinputlisting[style=customasm]{../../src/main/java/thewaypointers/trafficsimulator/common/helpers/RoadNetworkProvider.java}
\newpage
\subsection{common/helpers/SimpleStateChangeListener.java}
\lstinputlisting[style=customasm]{../../src/main/java/thewaypointers/trafficsimulator/common/helpers/SimpleStateChangeListener.java}
\newpage
\subsection{common/ILocation.java}
\lstinputlisting[style=customasm]{../../src/main/java/thewaypointers/trafficsimulator/common/ILocation.java}
\newpage
\subsection{common/ISimulationInputListener.java}
\lstinputlisting[style=customasm]{../../src/main/java/thewaypointers/trafficsimulator/common/ISimulationInputListener.java}
\newpage
\subsection{common/IStateChangeListener.java}
\lstinputlisting[style=customasm]{../../src/main/java/thewaypointers/trafficsimulator/common/IStateChangeListener.java}
\newpage
\subsection{common/IStateProvider.java}
\lstinputlisting[style=customasm]{../../src/main/java/thewaypointers/trafficsimulator/common/IStateProvider.java}
\newpage
\subsection{common/JunctionDTO.java}
\lstinputlisting[style=customasm]{../../src/main/java/thewaypointers/trafficsimulator/common/JunctionDTO.java}
\newpage
\subsection{common/JunctionLocationDTO.java}
\lstinputlisting[style=customasm]{../../src/main/java/thewaypointers/trafficsimulator/common/JunctionLocationDTO.java}
\newpage
\subsection{common/JunctionMoveResult.java}
\lstinputlisting[style=customasm]{../../src/main/java/thewaypointers/trafficsimulator/common/JunctionMoveResult.java}
\newpage
\subsection{common/JunctionTrafficLightsDTO.java}
\lstinputlisting[style=customasm]{../../src/main/java/thewaypointers/trafficsimulator/common/JunctionTrafficLightsDTO.java}
\newpage
\subsection{common/Lane.java}
\lstinputlisting[style=customasm]{../../src/main/java/thewaypointers/trafficsimulator/common/Lane.java}
\newpage
\subsection{common/MapDTO.java}
\lstinputlisting[style=customasm]{../../src/main/java/thewaypointers/trafficsimulator/common/MapDTO.java}
\newpage
\subsection{common/NodeDTO.java}
\lstinputlisting[style=customasm]{../../src/main/java/thewaypointers/trafficsimulator/common/NodeDTO.java}
\newpage
\subsection{common/RoadDTO.java}
\lstinputlisting[style=customasm]{../../src/main/java/thewaypointers/trafficsimulator/common/RoadDTO.java}
\newpage
\subsection{common/RoadLocationDTO.java}
\lstinputlisting[style=customasm]{../../src/main/java/thewaypointers/trafficsimulator/common/RoadLocationDTO.java}
\newpage
\subsection{common/RoadTrafficLightsDTO.java}
\lstinputlisting[style=customasm]{../../src/main/java/thewaypointers/trafficsimulator/common/RoadTrafficLightsDTO.java}
\newpage
\subsection{common/SimulationInputListener.java}
\lstinputlisting[style=customasm]{../../src/main/java/thewaypointers/trafficsimulator/common/SimulationInputListener.java}
\newpage
\subsection{common/TrafficLightColor.java}
\lstinputlisting[style=customasm]{../../src/main/java/thewaypointers/trafficsimulator/common/TrafficLightColor.java}
\newpage
\subsection{common/TrafficLightDTO.java}
\lstinputlisting[style=customasm]{../../src/main/java/thewaypointers/trafficsimulator/common/TrafficLightDTO.java}
\newpage
\subsection{common/TrafficLightSystemDTO.java}
\lstinputlisting[style=customasm]{../../src/main/java/thewaypointers/trafficsimulator/common/TrafficLightSystemDTO.java}
\newpage
\subsection{common/VehicleDTO.java}
\lstinputlisting[style=customasm]{../../src/main/java/thewaypointers/trafficsimulator/common/VehicleDTO.java}
\newpage
\subsection{common/VehicleListDTO.java}
\lstinputlisting[style=customasm]{../../src/main/java/thewaypointers/trafficsimulator/common/VehicleListDTO.java}
\newpage
\subsection{common/VehicleType.java}
\lstinputlisting[style=customasm]{../../src/main/java/thewaypointers/trafficsimulator/common/VehicleType.java}
\newpage
\subsection{common/WorldStateDTO.java}
\lstinputlisting[style=customasm]{../../src/main/java/thewaypointers/trafficsimulator/common/WorldStateDTO.java}
\newpage
\subsection{FirstVersionStarter.java}
\lstinputlisting[style=customasm]{../../src/main/java/thewaypointers/trafficsimulator/FirstVersionStarter.java}
\newpage
\subsection{gui/ControlPanel.java}
\lstinputlisting[style=customasm]{../../src/main/java/thewaypointers/trafficsimulator/gui/ControlPanel.java}
\newpage
\subsection{gui/ControlPanelDocumentListener.java}
\lstinputlisting[style=customasm]{../../src/main/java/thewaypointers/trafficsimulator/gui/ControlPanelDocumentListener.java}
\newpage
\subsection{gui/GuiController.java}
\lstinputlisting[style=customasm]{../../src/main/java/thewaypointers/trafficsimulator/gui/GuiController.java}
\newpage
\subsection{gui/JumpOutDialog.java}
\lstinputlisting[style=customasm]{../../src/main/java/thewaypointers/trafficsimulator/gui/JumpOutDialog.java}
\newpage
\subsection{gui/MainFrame.java}
\lstinputlisting[style=customasm]{../../src/main/java/thewaypointers/trafficsimulator/gui/MainFrame.java}
\newpage
\subsection{gui/MainFrameEventHandle.java}
\lstinputlisting[style=customasm]{../../src/main/java/thewaypointers/trafficsimulator/gui/MainFrameEventHandle.java}
\newpage
\subsection{gui/MapContainerPanel.java}
\lstinputlisting[style=customasm]{../../src/main/java/thewaypointers/trafficsimulator/gui/MapContainerPanel.java}
\newpage
\subsection{gui/MapPanel.java}
\lstinputlisting[style=customasm]{../../src/main/java/thewaypointers/trafficsimulator/gui/MapPanel.java}
\newpage
\subsection{gui/PanelMouseAction.java}
\lstinputlisting[style=customasm]{../../src/main/java/thewaypointers/trafficsimulator/gui/PanelMouseAction.java}
\newpage
\subsection{gui/Statistics.java}
\lstinputlisting[style=customasm]{../../src/main/java/thewaypointers/trafficsimulator/gui/Statistics.java}
\newpage
\subsection{gui/StatisticsPanel.java}
\lstinputlisting[style=customasm]{../../src/main/java/thewaypointers/trafficsimulator/gui/StatisticsPanel.java}
\newpage
\subsection{JunctionLocationTestStarter.java}
\lstinputlisting[style=customasm]{../../src/main/java/thewaypointers/trafficsimulator/JunctionLocationTestStarter.java}
\newpage
\subsection{RoadNetworkStarter.java}
\lstinputlisting[style=customasm]{../../src/main/java/thewaypointers/trafficsimulator/RoadNetworkStarter.java}
\newpage
\subsection{simulation/enums/NodeType.java}
\lstinputlisting[style=customasm]{../../src/main/java/thewaypointers/trafficsimulator/simulation/enums/NodeType.java}
\newpage
\subsection{simulation/enums/VehicleType.java}
\lstinputlisting[style=customasm]{../../src/main/java/thewaypointers/trafficsimulator/simulation/enums/VehicleType.java}
\newpage
\subsection{simulation/factories/GraphFactory.java}
\lstinputlisting[style=customasm]{../../src/main/java/thewaypointers/trafficsimulator/simulation/factories/GraphFactory.java}
\newpage
\subsection{simulation/factories/MapWorldStateFactory.java}
\lstinputlisting[style=customasm]{../../src/main/java/thewaypointers/trafficsimulator/simulation/factories/MapWorldStateFactory.java}
\newpage
\subsection{simulation/factories/VehicleFactory.java}
\lstinputlisting[style=customasm]{../../src/main/java/thewaypointers/trafficsimulator/simulation/factories/VehicleFactory.java}
\newpage
\subsection{simulation/factories/xml/models/HandlerXML.java}
\lstinputlisting[style=customasm]{../../src/main/java/thewaypointers/trafficsimulator/simulation/factories/xml/models/HandlerXML.java}
\newpage
\subsection{simulation/factories/xml/models/JunctionXML.java}
\lstinputlisting[style=customasm]{../../src/main/java/thewaypointers/trafficsimulator/simulation/factories/xml/models/JunctionXML.java}
\newpage
\subsection{simulation/factories/xml/models/MapXML.java}
\lstinputlisting[style=customasm]{../../src/main/java/thewaypointers/trafficsimulator/simulation/factories/xml/models/MapXML.java}
\newpage
\subsection{simulation/factories/xml/models/RoadXML.java}
\lstinputlisting[style=customasm]{../../src/main/java/thewaypointers/trafficsimulator/simulation/factories/xml/models/RoadXML.java}
\newpage
\subsection{simulation/models/graph/helper/DistributedRandomNumberGenerator.java}
\lstinputlisting[style=customasm]{../../src/main/java/thewaypointers/trafficsimulator/simulation/models/graph/helper/DistributedRandomNumberGenerator.java}
\newpage
\subsection{simulation/models/graph/helper/Node.java}
\lstinputlisting[style=customasm]{../../src/main/java/thewaypointers/trafficsimulator/simulation/models/graph/helper/Node.java}
\newpage
\subsection{simulation/models/graph/helper/RoadEdge.java}
\lstinputlisting[style=customasm]{../../src/main/java/thewaypointers/trafficsimulator/simulation/models/graph/helper/RoadEdge.java}
\newpage
\subsection{simulation/models/graph/helper/TrafficLightNode.java}
\lstinputlisting[style=customasm]{../../src/main/java/thewaypointers/trafficsimulator/simulation/models/graph/helper/TrafficLightNode.java}
\newpage
\subsection{simulation/models/interfaces/IVehicle.java}
\lstinputlisting[style=customasm]{../../src/main/java/thewaypointers/trafficsimulator/simulation/models/interfaces/IVehicle.java}
\newpage
\subsection{simulation/models/managers/VehicleManager.java}
\lstinputlisting[style=customasm]{../../src/main/java/thewaypointers/trafficsimulator/simulation/models/managers/VehicleManager.java}
\newpage
\subsection{simulation/models/VehicleMap.java}
\lstinputlisting[style=customasm]{../../src/main/java/thewaypointers/trafficsimulator/simulation/models/VehicleMap.java}
\newpage
\subsection{simulation/models/vehicles/Car.java}
\lstinputlisting[style=customasm]{../../src/main/java/thewaypointers/trafficsimulator/simulation/models/vehicles/Car.java}
\newpage
\subsection{simulation/models/vehicles/EmergencyService.java}
\lstinputlisting[style=customasm]{../../src/main/java/thewaypointers/trafficsimulator/simulation/models/vehicles/EmergencyService.java}
\newpage
\subsection{simulation/Simulation.java}
\lstinputlisting[style=customasm]{../../src/main/java/thewaypointers/trafficsimulator/simulation/Simulation.java}
\newpage
\subsection{StateProviderController.java}
\lstinputlisting[style=customasm]{../../src/main/java/thewaypointers/trafficsimulator/StateProviderController.java}
\newpage
\subsection{TrafficSimulatorStarter.java}
\lstinputlisting[style=customasm]{../../src/main/java/thewaypointers/trafficsimulator/TrafficSimulatorStarter.java}
\newpage
\subsection{utils/Angle.java}
\lstinputlisting[style=customasm]{../../src/main/java/thewaypointers/trafficsimulator/utils/Angle.java}
\newpage
\subsection{utils/FloatPoint.java}
\lstinputlisting[style=customasm]{../../src/main/java/thewaypointers/trafficsimulator/utils/FloatPoint.java}
\newpage
\subsection{utils/Pair.java}
\lstinputlisting[style=customasm]{../../src/main/java/thewaypointers/trafficsimulator/utils/Pair.java}
\newpage
\subsection{utils/Rotation.java}
\lstinputlisting[style=customasm]{../../src/main/java/thewaypointers/trafficsimulator/utils/Rotation.java}
\newpage
\subsection{utils/SpeedConvert.java}
\lstinputlisting[style=customasm]{../../src/main/java/thewaypointers/trafficsimulator/utils/SpeedConvert.java}
\newpage
\subsection{utils/VehicleSpawnRatio.java}
\lstinputlisting[style=customasm]{../../src/main/java/thewaypointers/trafficsimulator/utils/VehicleSpawnRatio.java}


\section{Test source code}


\newpage
\subsection{tests/common/helpers/FirstVersionProviderTest.java}
\lstinputlisting[style=customasm]{../../src/test/java/thewaypointers/trafficsimulator/tests/common/helpers/FirstVersionProviderTest.java}
\newpage
\subsection{tests/common/JunctionLocationDTOTest.java}
\lstinputlisting[style=customasm]{../../src/test/java/thewaypointers/trafficsimulator/tests/common/JunctionLocationDTOTest.java}
\newpage
\subsection{tests/common/MapDTOTest.java}
\lstinputlisting[style=customasm]{../../src/test/java/thewaypointers/trafficsimulator/tests/common/MapDTOTest.java}
\newpage
\subsection{tests/FirstVersionStarterTest.java}
\lstinputlisting[style=customasm]{../../src/test/java/thewaypointers/trafficsimulator/tests/FirstVersionStarterTest.java}
\newpage
\subsection{tests/gui/GuiControllerTest.java}
\lstinputlisting[style=customasm]{../../src/test/java/thewaypointers/trafficsimulator/tests/gui/GuiControllerTest.java}
\newpage
\subsection{tests/JunctionLocationTestStarterTest.java}
\lstinputlisting[style=customasm]{../../src/test/java/thewaypointers/trafficsimulator/tests/JunctionLocationTestStarterTest.java}
\newpage
\subsection{tests/RoadNetworkStarterTest.java}
\lstinputlisting[style=customasm]{../../src/test/java/thewaypointers/trafficsimulator/tests/RoadNetworkStarterTest.java}
\newpage
\subsection{tests/TrafficSimulatorStarterTest.java}
\lstinputlisting[style=customasm]{../../src/test/java/thewaypointers/trafficsimulator/tests/TrafficSimulatorStarterTest.java}
\newpage
\subsection{tests/utils/FloatPointTest.java}
\lstinputlisting[style=customasm]{../../src/test/java/thewaypointers/trafficsimulator/tests/utils/FloatPointTest.java}
\newpage
\subsection{utils/AngleTest.java}
\lstinputlisting[style=customasm]{../../src/test/java/thewaypointers/trafficsimulator/utils/AngleTest.java}


\end{document}
